%PREAMBULO
%Tipo de reporte
\documentclass[12pt]{report}
%Diseño del documento
%\usepackage[total={18cm,21cm},top=2cm, left=2cm]{geometry}
%Permite incluir simbolos de la AMS
\usepackage{amsmath,amssymb,amsfonts}
%Paquete para incluir acentos
\usepackage[utf8]{inputenc}
\usepackage[T1]{fontenc}
%Paquete para manejo de gráficos y Figuras
\usepackage{graphicx}
%Paquete para documentos en español
\usepackage[spanish]{babel}
%Paquete para las letras
\usepackage{lettrine}
%Paquete para enlazar
\usepackage [hidelinks]{hyperref}
%Paquete para la Cabecera y el pie de pagina
\usepackage{fancyhdr}
%Paquete para formatear codigo C
\usepackage{listings}
%Paquete para usar marcos en el codigo fuente
\usepackage{fancybox, graphicx}
%Paquete para manejar la posicion de las imagenes
\usepackage{float}
%Para numerar ejemplos y definiciones
\newtheorem{ejemplo}{{\it Ejemplo}}[chapter]
\newtheorem{defi}{{\it Definición}}[chapter]
%Paquete para pseudocodigos
%\usepackage{algorithm2e}
%%Para hipervinculos
\usepackage{hyperref}
%% PONEMOS EN FORMA HORIZONTAL
\usepackage{lscape}
%% PARA LAS LIGAS DE INTERNET
\usepackage[usenames]{color}
%APENDICES
\usepackage{appendix}
%%% PARA AGREGAR PDF'S%%%%%
%%%PAQUETE PARA TRATAR CON TABLAS GRANDES%%%
\usepackage{longtable}
\usepackage{multirow,array}


%%%%NUMERAMOS LAS ECUACIONES 
\numberwithin{equation}{section}
%Cuerpo de documento
\begin{document}
% ESTILO DE CABECERAS
\pagestyle{fancy}
\renewcommand{\chaptermark}[1]{\markboth{\MakeUppercase{\chaptername}\ \thechapter.\ #1}{}}
%Grosor de la linea de cabeceras
\renewcommand{\headrulewidth}{0.2pt} 
\pagenumbering{Roman} 

\newcommand{\plan}{Plan del Proyecto}
\newcommand{\iso}{ISO/IEC 29110 Perfil B\'asico}
\newcommand{\isoP}{ISO/IEC 29110 Perfil B\'asico }
%\newcommand{\plan}{Plan del Proyecto}
\newcommand{\planE}{Plan del Proyecto }
\newcommand{\implementacion}{Implementaci\'on de Software}
\newcommand{\reporte}{Reporte de seguimiento }



\pagenumbering{arabic}
\setcounter{page}{26}


\newpage


\centering\section{Plan del Proyecto 6 iteraci\'on}\label{Plan del Proyecto5}
%\large{1.Introducci\'on}\\
%\vspace{1em}
%\small{\textit{[La introducci\'on ofrece un resumen del contenido del documento y del \'area de estudio del proyecto.}]}
%\vspace{2em}
%
%
%\large{2.Enunciado de Trabajo}\\
%\vspace{1em}
%
%\large{2.1 Objetivos Generales}\\
%\vspace{1em}
%\small{\textit{[En esta sección se documenta el enunciado de trabajo del proyecto en cuestión,  se consideran las metas que se esperan alcanzar con los productos de software terminados.}]}
%\vspace{2em}
\begin{flushleft}
\large{3.Objetivos y Tareas}\\
\vspace{1em}


\large{3.1 Objetivos}\\
\vspace{1em}
%\small{\textit{[Objetivos que se desean alcanzar al final del tiempo asignado al \plan, estos pueden abarcar desde  investigaci\'on hasta la construcci\'on de software o entrega del producto.}]}
%\vspace{2em}

\begin{longtable}[H]{|m{0.5cm}|m{7cm}|m{5.5cm}|}
\hline
\small{\textbf{ID}} & \small{\textbf{Objetivos}} & \small{\textbf{Estado del Objetivo}} \\
\hline \hline
\endfirsthead
\hline
\small{\textbf{ID}} & \small{\textbf{Objetivos}} & \small{\textbf{Estado del Objetivo}} \\
\hline \hline
\endhead
\hline
\endfoot

\endlastfoot
\textbf{1}  & \small{Inicio de pruebas de integraci\'on } & \small{\textit{Iniciadas}}\\
\hline 
\textbf{2}  & \small{Investigaci\'on de métodos numéricos para la soluci\'on de sistemas de ecuaciones lineales} & \small{\textit{Iniciado}}\\
\hline 
\textbf{3}  & \small{Construcci\'on de los métodos numéricos para la soluci\'on de sistemas de cuaciones lineales} & \small{\textit{Iniciado}}\\
\hline
\hline
\caption{{\footnotesize Lista de objetivos a completar.}}
\label{tabla: TABLA CE quinta plan}
\end{longtable}


\large{3.2.Tareas}\\
%\vspace{1em}
%\small{\textit{[Las tareas registradas pueden ayudar al cumplimiento de los objetivos.]}}
%\vspace{2em}
\begin{table}[H]
\centering
\begin{tabular}{|m{0.5cm}|m{0.7cm}|m{2.5cm}|m{3cm}|m{2.5cm}|m{2.5cm}|}
\hline
\textbf{ID} & \multicolumn{2}{|l|}{\textbf{Funcionalidad o Actividad}} & \textbf{Responsable} & \textbf{Fecha de Entrega} & \textbf{Estado (Iniciado/ Terminado/ Pendiente)}\\
\hline \hline
1 & \multicolumn{5}{|l|}{\textit{Pruebas}}\\
\hline
  & \textit{1.1} & \small{Definir pruebas para el almacenamiento de matrices} & \small{ET} & \small{\textit{9-09-2016}} &  \small{Iniciado}\\ 
\hline
  & \textit{1.2} &\small{Definir pruebas para el almacenamiento de vectores.} & \small{ET}& \small{\textit{09-09-2016}} &  \small{Iniciado}\\
\hline
 & \textit{1.3} &\small{Realizar pruebas.} & \small{ET}& \small{\textit{15-08-2016}} &  \small{Iniciado}\\ 
\hline
 & \textit{1.4} &\small{Realizar pruebas de integraci\'on de los paquetes de vetores y matrices, con el paquete CVFEM} & \small{ET}& \small{\textit{09-09-2016}} &  \small{Iniciado}\\ 
\hline
 2 & \multicolumn{5}{|l|}{\textit{Investigaci\'on}}\\
\hline
 & \textit{2.1} &\small{Investigaci\'on de métodos n\'umericos} & \small{ET}& \small{\textit{21-09-2016}} &  \small{Pendiente}\\ 
  & \textit{2.2} & \small{Investigaci\'on de métodos n\'umericos iterativospara el esquema de almacenamiento} &\small{ET} & \small{\textit{21-09-2016}} &  \small{Pendiente} \\ 
\hline
 3 & \multicolumn{5}{|l|}{\textit{Construcci\'on}}\\
\hline
  & \textit{3.1} & \small{Implementaci\'on de m\'etodos num\'ericos en el esquema de almacenamiento} & \small{ET} & \small{\textit{30-09-2016}} &  \small{Pendiente} \\ 
\hline

\end{tabular}

\end{table}




\large{4.Riesgos}\\
%\vspace{1em}
%\small{\textit{[Listado de los riesgos identificados que podr\'ian afectar el desarrollo del proyecto en la iteraci\'on actual en caso de ocurrencia.]}}
%\vspace{2em}


\begin{table}[H]
\begin{tabular}{|m{2cm}|m{4cm}|m{4.5cm}|m{1.5cm}|m{2cm}|}
\hline 
\textbf{Nombre } & \textbf{Descripci\'on} & \textbf{Plan de contingencia} & \textbf{Impacto} & \textbf{Estado del riesgo}  \\
\hline
\hline
\small{--} & \small{--}& \small{--}  & \small{--} & \small{--}\\
\hline
\end{tabular}
%\caption{{\footnotesize  Tabla de riesgos nuevos y que se han manifestado.}}
\label{tabla: TABLA CE de nuevos riesgos}
\end{table}

\end{flushleft}

%%%%%%%%%%%%%%%%%%%%%%%%%%%%%%%%%%%%%%%%%%%%%%%%%%%%Nuevo doc%%%%%%%%%%%%%%%%%%%%%%%%%%%%555555
\newpage

\centering \section{Reporte de Seguimiento} 
%\vspace{1em}


\begin{tabular}{m{7cm} m{8cm}}
\small{ \textbf{Proyecto:} \scriptsize{\textit{[Simulaci\'on de flujo de medios porosos usando CVFE]}}} & \small{\textbf{Periodo a Reportar:} \scriptsize{01-09-2016.] al [19-09-2016]}}
\end{tabular}

\begin{flushleft}

\large{1.Reporte de Actividades }\\
%\vspace{1em}
%\small{\textit{[Listado de todas las actividades y objetivos realizados por el equipo durante el periodo de tiempo reportado.]}}
%\vspace{2em}


\begin{table}[H]
\begin{tabular}{|m{2cm}|m{4.5cm}|m{1.5cm}|m{1.5cm}|m{4.5cm}|}
\hline
\small{\textbf{Integrante}} &\small{ \textbf{Actividad u Objetivo}} &\small{ \textbf{Tiempo Total}} & \small{\textbf{Estado} }& \small{\textbf{Observaciones}}\\
\hline \hline
ET & Realizaci\'on de pruebas para el almacenamiento de matrices  & 8 hr & Finalizado & \small{Las pruebas mostraron un buen funcionamiento, aun faltan ciertos elementos por probar.}\\
\hline
ET & Pruebas para el almacenamiento de vectores  & 8 hr & Finalizado & \small{Falta definir ciertos elementos a corregir, las pruebas han mostrado que realiza los c\'alculos de manera correcta, el modulo se tiene en su mayoria terminado.}\\
\hline
ET & Pruebas de integraci\'on  & 8 hr & Proceso & \small{Las pruebas realizadas fueron definidas como pruebas de caja negra, aun existen elementos faltantes para establecer la finalizaci\'on de las pruebas.}\\
\hline
ET & Investigaci\'on de métodos num\'ericos  & 30 hr & Proceso & \small{La investigaci\'on sigue su curso, aun se tienen dudas sobre ciertos elementos.}\\
\hline
ET & Construcci\'on de la implementaci\'on de métodos num\'ericos.  & -- & Pendiente & --\\
\hline
\multicolumn{2}{|c|}{Total de Horas trabajadas} & 54 hr  & & \\
\hline 
\end{tabular}
%\caption{{\footnotesize  Tabla de los Objetivos y actividades a realizar.}}
\label{tabla: TABLA CE Actividades}
\end{table}



\large{2.Tareas finalizadas }\\
%\vspace{1em}
%\small{\textit{[Registro de tareas que han finalizado en el momento de la realizaci\'on del reporte.]}}
%\vspace{2em}

\begin{longtable}[H]{|m{5cm}|m{3cm}|m{3cm}|m{3cm}|} 
\hline
\small{\textbf{Actividad y/o tarea }} &\small{ \textbf{Integrante}} & \small{\textbf{Fecha de entrega}} & \small{\textbf{Fecha real}}\\
\hline \hline
\endfirsthead

\hline
\small{\textbf{Actividad y/o tarea }} &\small{ \textbf{Integrante}} & \small{\textbf{Fecha de entrega}} & \small{\textbf{Fecha real}}\\
\hline \hline
\endhead
\hline
\endfoot

\endlastfoot
\hline
\hline
\small{} & \small{Realizaci\'on de pruebas para el almacenamiento de matrices} &\small{ \textit{[09-09-2016]}} & \small{\textit{[09-09-2016]}}\\
\hline
\small{ET} &\small{Pruebas para el almacenamiento de vectores} &\small{ \textit{[09-09-2016]}} & \small{\textit{[09-09-2016]}}\\
\hline
\hline
%\caption{Roles}
%\label{tabla:TAreasProy}
\end{longtable}

\newpage

\large{3.Productos }
%\vspace{1em}
%\small{\textit{[Listado de todos los productos de software  realizados por el equipo y su estado actual.]}}
%\vspace{2em}
\begin{table}[H]
\begin{tabular}{|m{0.5cm}|m{7.5cm}|m{6cm}|}
\hline 
\textbf{ID } & \textbf{Producto} & \textbf{Estado} \\
\hline
\hline
 1 & \small{Clase malla} & \small{Finalizado}\\
\hline
 2 & \small{CVFEM} & \small{En Proceso}\\
\hline
\end{tabular}
%\caption{{\footnotesize  Tabla de Productos.}}
\label{tabla: TABLA CE Productos}
\end{table}



\large{4.Cambios}\\
%\vspace{1em}
%\small{\textit{[Registro de los cambios necesarios tanto en documentaci\'on como en los productos de software generados.]}}
%\vspace{2em}

\begin{table}[H]
\begin{tabular}{|m{0.5cm}|m{2cm}|m{4.5cm}|m{3.5cm}|m{3.5cm}|}
\hline 
\textbf{ID} & \textbf{Producto} & \textbf{Descripci\'on} & \textbf{Solicitante} & \textbf{Estado}  \\
\hline
\hline
1 & \small{\'Algebra Lineal}  & \small{Problemas de memoria, se presenta el mensaje de Segmention full} &\small{ET} & \small{Proceso}\\
\hline
2 & \small{\'Algebra Lineal}  & \small{Se desea poder convertir el almacenamiento de matrices de COO a CSR} &\small{ET} & \small{Proceso}\\
\hline
3 & \small{\'Algebra Lineal}  & \small{Liberaci\'on correcta de memoria} &\small{ET} & \small{Proceso}\\
\hline
4 & \small{Vectores}  & \small{Problemas de liberaci\'on de memoria} &\small{ET} & \small{Proceso}\\
\hline
5 & \small{Vectores}  & \small{Se debe de modificar la codificaci\'on realizada siguiendo la normatividad propuesta.} &\small{ET} & \small{Proceso}\\
\hline
6 & \small{Integraci\'on}  & \small{Existe un mal funcionamiento en el c\'alculo de los coeficientes $T_{i,j}$ y $T_{j,k}$ de los elementos vecinos, ademas de que se ha observado que la matriz no se construye correctamente.} &\small{ET} & \small{Proceso}\\
\hline
\end{tabular}
%\caption{{\footnotesize  Tabla que lleva el registro de cambios a realizar.}}
\label{tabla: TABLA CE Cambios Seg}
\end{table}

\newpage

\large{5.Riesgos}\\
%\vspace{1em}
%\small{\textit{[Listado de los riesgos identificados que podr\'ian afectar el desarrollo del proyecto en la iteraci\'on actual en caso de ocurrencia.]}}
%\vspace{2em}


\begin{table}[H]
\begin{tabular}{|m{2cm}|m{4cm}|m{4cm}|m{2cm}|m{2cm}|}
\hline 
\textbf{Nombre } & \textbf{Descripci\'on} & \textbf{Plan de contingencia} & \textbf{Impacto} & \textbf{Estado del riesgo}  \\
\hline
\hline
\small{--} &\small{--} & \small{--} & --   & -- \\
\hline
\end{tabular}
%\caption{{\footnotesize  Tabla de riesgos nuevos y que se han manifestado.}}
\label{tabla: TABLA CE de nuevos riesgos Seg}
\end{table}

\large{6.Resumen}\\
%\vspace{1em}
%\small{\textit{[Es conveniente resumir las mediciones tomadas en el periodo de tiempo que abarca el seguimiento, para ello se proponen tablas que incluyan la informaci\'on m\'as relevante de una manera breve y clara.]}}
%\vspace{2em}

\begin{table}[H]
\begin{tabular}{|m{5cm}|m{5cm}|m{5cm}|}
\hline
\textbf{Tareas} & \textbf{Cambios} & \textbf{Riesgos}\\
\hline \hline 
A tiempo: 2 & Solicitados: 6 & Encontrados 0 \\
\hline
Retrasadas: 2 & Rechazados: 0 & Resueltos 0 \\
\hline
Adelantadas:0  & Realizados: 0  & Postergados 0 \\
\hline
Postergadas: 1 & Postergados:  0 & \\
\hline
\end{tabular}
%\caption{{\footnotesize  Resumen de actividades.}}
\label{tabla: TABLA CE Resumen}
\end{table}

\end{flushleft}

\newpage
%%%%%%%%%%%%%%%%%%%%%%%%%%%%%%%%%%%%%%%%%%%%%%%%%%%%%%%%%%%nuevo documento%%%%%%%%%%%%%%%%%%%%%%%%%%%%%%%%%%%%%%%%%

\centering \section{Cierre del Plan}\label{CE Cierre}
%\vspace{1em}

\begin{flushleft}

\textbf{1.Resumen}\\
%\vspace{1em}
%\scriptsize{\textit{[Es conveniente resumir las mediciones registradas durante la ejecuci\'on del \plan, para ello se proponen tablas que incluyan la informaci\'on m\'as relevante de una manera breve y clara. Los objetivos se registran para tener una perspectiva del avance del proyecto, con el objetivo de tomar decisiones para concluir el proyecto.]}}
%\vspace{2em}

\begin{table}[H]
\centering
\begin{tabular}{|m{5cm}|m{5cm}|}
\hline
\textbf{Productos} & \textbf{Cambios}\\
\hline \hline
\small{A tiempo: 2 }&\small{ Solicitados: 6}\\
\hline
\small{Retrasados: 1}& \small{Rechazados:0}\\
\hline
\small{Adelantados: 0}& \small{Realizados:5}\\
\hline
\small{Postergados: 1}& \small{Postergados:1}\\
\hline
\end{tabular}
\end{table}

\begin{table}[H]
\centering
\begin{tabular}{|m{5cm}|m{5cm}|}
\hline
\textbf{Riesgos} & \textbf{Actividades u Objetivos}\\
\hline \hline
\small{Encontrados: 0}& \small{Postergados: 1}\\
\hline
\small{Resueltos: 0} & \small{Alcanzados: 2}\\
\hline
%Postergados: & \\
%\hline
\end{tabular}
%\caption{{\footnotesize  Resumen de actividades Cierre.}}
\label{tabla: TABLA CE ResumenCierre}
\end{table}

\end{flushleft}

\begin{flushleft}

\textbf{2.Reporte de productos.}\\
%\vspace{1em}
%\scriptsize{\textit{[Se realiza un recuento de los productos, defectos y su estado al final de la ejecuci\'on del \plan.]}}
%\vspace{2em}

\begin{table}[H]
\begin{tabular}{|m{4cm}|m{3cm}|m{3cm}|m{5cm}|}
\hline
\textbf{Nombre del producto} & \multicolumn{2}{c|}{\textbf{Cambios}}  
 & \textbf{Estado del producto}\\
 \hline
 & \textbf{Encontrados} &\textbf{Corregidos} & \\
\hline
\small{Almacenamiento del vector} & 2 & 2 &\scriptsize{\textit{Finalizado}} \\
\hline 
\small{Esquemas de matrices dispersas}& 4 & 3 & \scriptsize{\textit{{Postergado}}} \\
\hline
\small{CVFEM}  & 0 & 0 &\scriptsize{\textit{Postergado}} \\
\hline
\small{Malla} & 0 & 0 &\scriptsize{\textit{Finalizado}} \\
\hline
\end{tabular}
%\caption{{\footnotesize  Tabla de los Objetivos y actividades a realizar.}}
\label{tabla: TABLA Actividades CE }
\end{table}

\newpage

\textbf{3.Retroalimentaci\'on.}\\
%\vspace{1em}
%\scriptsize{\textit{[Listar mejoras y sugerencias del proceso seguido, listar las tareas que se pretenden realizar en la siguiente iteraci\'on anexando las tareas del \plan anterior que aun est\'an en proceso y los acuerdos que el equipo ha decidido realizar para el mejoramiento del trabajo.]}}
%\vspace{2em}

\small{Los productos codificados han mostrado ciertos elementos a cambiar, pero se ha observado que los cambios fueron mas en el uso de memoria, se ha platicado ya con el administrador, y se han envido el trabajo realizado para su revisi\'on, también se ha solicitado el incrementar el trabajo, para logra terminar a tiempo, este mes hubo muchos días feriados, lo cual produjo una disminuci\'on en el trabajo realizado, el siguiente mes se ha propuesto el aumentar las horas trabajadas.}\\
\vspace{1em}

\small{\textbf{Tareas por terminar.}\\

\begin{itemize}
\item Investigaci\'on de los métodos numéricos para resolver sistemas de ecuaciones lineales.
\item Programaci\'on de los m\'etodos num\'ericos para resolver sistemas de ecuaciones lineales.
\end{itemize}}


%\textbf{4.Comentarios}\\

%\small{} 
\end{flushleft}

\end{document}